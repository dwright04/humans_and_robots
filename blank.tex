\documentclass[a4paper,fleqn,usenatbib]{mnras}

\usepackage{graphicx}	% Including figure files
\usepackage{amsmath}	% Advanced maths commands
\usepackage{amssymb}	% Extra maths symbols

\title[Human and machine classifications]{A transient search using combined human and machine classifications}

\author[D. Wright, C. Lintott et al.]{
Daryll Wright,$^{1}$\thanks{E-mail: daryll@zooniverse.org}
Chris Lintott,$^{2}$

$^{1}$Department of Physics, University of Oxford, Denys Wilkinson Building, Keble Road, Oxford, OX1 3RH
}

\date{Accepted XXX. Received YYY; in original form ZZZ}

\pubyear{2016}

\begin{document}
\label{firstpage}
\pagerange{\pageref{firstpage}--\pageref{lastpage}}
\maketitle

% Abstract of the paper
\begin{abstract}
Large modern surveys require efficient review of data in order to find transient sources such as supernovae, and to distinguish such sources from artefacts, noise and so on. Much effort has been put into the development of automatic algorithms, but surveys still rely on human review of targets. This paper presents an integrated system for the identification of supernovae in data from PanSTARRS, combining classifications from a citizen science project including volunteers with those from a convolutional neural network. This work represents the first time such a system has been deployed on real astronomical data, and we show that the combination of the two methods outperforms either one used individually. This result has important implications for the future development of transit searches, especially in the era of LSST and other large-throughput surveys. 
\end{abstract}

\begin{keywords}
keyword1 -- keyword2 -- keyword3
\end{keywords}

%%%%%%%%%%%%%%%%%%%%%%%%%%%%%%%%%%%%%%%%%%%%%%%%%%

%%%%%%%%%%%%%%%%% BODY OF PAPER %%%%%%%%%%%%%%%%%%

\section{Introduction}


\begin{figure*}
   %\vspace{200pt}
   \includegraphics[width=164mm]{figs/.png}
   \caption{} 
   \label{fig:machine_dist} 
\end{figure*}


\bsp	% typesetting comment
\label{lastpage}
\end{document}